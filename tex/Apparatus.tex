\subsection{Apparatus}
The coaxial cable being used in the experiment is 
bound in a two boxes. An simplified example of one is as in fig \ref{fig:coaxbox}.
Each box has 12 lengths of cable at $(16.19 \pm 0.03)$m each, as specified on the outside
of the box.

\begin{figure}[htbp]
\centering
\includegraphics{pics/coaxbox.pdf}
\caption{Simplified diagram of coaxial cable contained in a box, each box has
 12 lengths (not shown separately) \label{fig:coaxbox}}
\end{figure}

A circuit is set up using short lengths of coaxial cables to connect components
as in fig \ref{fig:circuit}. The short lengths of connecting cable are insignificant
compared to the amount of cabel being inspected. The terminating load from fig \ref{fig:circuit} on the end of the 
coaxial cable is $Z_l = 50\Omega$

\begin{figure}[htbp]
\centering
\includegraphics[width=0.45\textwidth]{pics/Circuit.pdf}
\caption{The circuit layout to find the transmission velocity) \label{fig:circuit}}
\end{figure}

The signal generator\footnote{R.R signal Generator Model RD-1U PY3404}
 is set to the values described in table \ref{tab:pulse}
The Oscilloscope\footnote{Tektronic TDS 2002B Two Channel Digital Oscilloscope}
 is a modern digital and can display two channels. One channel for
before, and the other after, the long length of coaxial cable.

All cables are in good condition and the connection between the components showed
no problems and appeared to be well fitting.

\begin{table}[htbp]
\centering
\begin{tabular}{l|l}
Period & $10\mu\text{s} \times 2.5 \rightarrow 25\mu\text{x}$  \\
\hline Delay & $1\mu\text{s} \times 2.0 \rightarrow 2\mu\text{x}$ \\
\hline Width & $0.1\mu\text{s} \times 7.0 \rightarrow 0.1\mu\text{x}$
\end{tabular}
\caption{Settings of the Pulse generator \label{tab:pulse}}
\end{table}
