\subsection{Theory \label{theory}}
The transmission velocity for a pulse is the standard velocity equation \ref{speed}
\begin{align}
&v_{\text{average}} = \frac{\Delta X}{\Delta t}   \label{speed}\\ 
\Rightarrow &v_{\text{transmission}} = \frac{l}{t} \label{transmissionvelocity}
\end{align}
Where $\Delta X$ and $l$ are the length of the cable, and
$\Delta t$ or $t$ is the time difference between the object (or signal in this case) between the beginning and end
of the length.

The \index{Velocity}velocity of a pulse is theoretically given by given by:
\begin{equation}
v = \frac{C}{\sqrt{\epsilon_r \mu_r}}\label{speed3}
\end{equation}
Where $C$ is the speed of light in a vacuum, $(\approx 3\times 10 ^8)$.
and $\epsilon_r \mu_r$ are values associated with physical of the material.
Using equation (\ref{transmissionvelocity}) and (\ref{speed3})
there is an equation from a experimentally measured $v$ to find $\epsilon_r$ assuming $\mu_r =1$:
\begin{equation}
\epsilon_r = \frac{C^2}{v^2} \label{er}
\end{equation}


