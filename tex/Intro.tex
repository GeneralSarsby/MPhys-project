\section{Introduction}
This report details how a signal down an coaxial cable travels. In particular,
the velocity of the wave pulse, the characteristic impedance of the cable (the same as
the wave impedance) and the attenuation of the wave.

The attenuation coefficient quantitatively describes how easy it is for a signal
to penetrate a material. Large values imply great ``attenuation'' (weakening) whereas
small values mean the material is quite transparent.

\subsection{Coaxial Cable}
\index{Coaxial Cable}Coaxial Cable is a multi-layered electrical cable as shown in fig \ref{fig:coax}.
There is an inner conductor surrounded by a dielectric insulating layer.
The cable is protected by a surrounding conducting shield and plastic sheath.

The signal propagates in the electromagnetic field in the dielectric insulator
between the inner and outer conductors.

The name `coaxial' is because the inner and outer conductor have the same geometric axis.

\begin{figure}[htbp]
\centering
\includegraphics[width=0.4\textwidth]{pics/coax.pdf}
\caption{Diagram of a Coaxial Cable \label{fig:coax}}
\end{figure}

\subsection{Uses of the Investigation}
Coaxial cable is used for transmitting radio frequency signals e.g. in connecting radio transmitters and receivers to antennas.
Other uses include network connections for computers. For electrical circuits that deal with wave pulses, the cable characteristics are
important consideration for the theory for the design.

In Telecommunication and networking to be able to send signals, or messages, with the
for maximum clarity and throughput (pulses/time) is useful.
This report would give some incite to technical limitations on what is possible.

\subsection{History}
Coaxial cable was invented by Oliver Heaviside \cite{OliverHeav} gaining patent no. 1,407 in 1880.
Heaviside (18 May 1850 -- 3 February 1925) was an English self taught engineer,
mathematician and physicist who coined many of electrical terms used today
including `impedance' and `inductance'\cite{oliver}.



